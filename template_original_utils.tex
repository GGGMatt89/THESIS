\begin{figure}[H]
\centering
	\includegraphics[width=0.9\textwidth]%
	{03_GraphicFiles/CowLickingNose.jpg}%
\caption[A cow]{A cow licking its nose. Usage with permission of the photographer \textsc{Nicole Barth}, taken from \url{www.flickr.com/photos/46311827@N07/14885545396}.}
\label{fig:CowLickingNose}
\end{figure}

In \figurename~\ref{fig:CowLickingNose}\myMarginnote{Reference to a figure} you see a cow that is licking its nose. The picture was taken by Nicole Barth on 11.08.2014 using a Canon EOS 500D. The original file has a resolution of $4247 \times 2831$ pixels.

\begin{figure}[H]
\centering
\begin{tikzpicture}
\begin{axis}[
axis lines = middle,
enlargelimits = true,
xlabel = {$x$},
ylabel = {$y$},
trig format plots = rad,
width = 0.9\textwidth,
height= 80mm,
title style={font=\bfseries,align=center,text width=0.7\textwidth},
title = {Example Diagram with a Line Break in the Title (using the \texttt{text width} option in the \texttt{title style})},
]
\addplot[myColorMainA,domain=0:9, line width=1pt, smooth]
{0.2*x^2};
\addplot[myColorMainB,domain=0:9, line width=1pt, smooth]
{5*sin(x)};
\end{axis}
\end{tikzpicture}
\caption[Scientific diagram]{A scientific diagram using the \texttt{pgfplots} package by \textsc{Christian Feuersaenger} using the same colors which are also used for the layout.}
\label{fig:ScientificDiagram}
\end{figure}


\begin{table}[H]
\centering
\begin{tabular}{@{}llr@{}} \toprule
\multicolumn{2}{c}{Item} \\ \cmidrule(r){1-2}
Animal & Description & Price (\$)\\ \midrule
Gnat & per gram & 13.65 \\
& each & 0.01 \\
Gnu & stuffed & 92.50 \\
Emu & stuffed & 33.33 \\
Armadillo & frozen & 8.99 \\ \bottomrule
\end{tabular}
\caption[Small table]{A small table created with the \texttt{booktabs} package (example taken from the package documentation).}
\end{table}



\blinditemize

\blindenumerate

\blindmathpaper
