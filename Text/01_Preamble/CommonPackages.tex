% Input encoding is 'latin1' (Latin 1 - also known as ISO-8859-1)
% CTAN: http://www.ctan.org/pkg/inputenc
%
% A newer package is available - you may look into:
% \usepackage[x-iso-8859-1]{inputenc}
% CTAN: http://www.ctan.org/pkg/inputenx
\usepackage[latin1]{inputenc}%OK

% Font Encoding is 'T1' -- important for special characters such as Umlaute � or � and special characters like � (enje)
% CTAN: http://www.ctan.org/pkg/fontenc
\usepackage[T1]{fontenc, url}%OK
\usepackage[bitstream-charter]{mathdesign}

% Language support for 'english' (alternative 'ngerman' or 'french' for example)
% CTAN: http://www.ctan.org/pkg/babel
\usepackage[english]{babel} %OK
\usepackage[babel]{csquotes}
\usepackage{hyphenat}
\hyphenation{tem-pe-ra-tu-re}
\hyphenation{re-la-ti-ve}

% For nice ordering numbers 
\usepackage[super]{nth}

% Doing calculations with LaTeX units -- needed for the vertical line in the footer
% CTAN: http://www.ctan.org/pkg/calc
\usepackage{calc}%OK

% Extended graphics support
% There is also a package named 'graphics' - watch out!
% CTAN: http://www.ctan.org/pkg/graphicx
\usepackage{graphicx}%OK

% Extendes support for floating objects (tables, figures), adds the [H] placing option (\begin{figure}[H]) which palces it "Here" (without any doubt).
% CTAN: http://www.ctan.org/pkg/float
\usepackage{float}

%for rotation of graphic environments (tables in particular)
\usepackage{pdflscape}
\usepackage{afterpage}
\usepackage{multirow}

% Support for captions, subcaptions and subfigures
\usepackage[margin=20pt]{subcaption}
\usepackage{caption}
\captionsetup{justification=justified, format=hang}

% Extended color support
% I use the command \definecolor for example.
% Option 'Table': Load the colortbl package, in order to use the tools for coloring rows, columns, and cells within tables.
% CTAN: http://www.ctan.org/pkg/xcolor
\usepackage[table]{xcolor}

% Nice tables
% CTAN: http://www.ctan.org/pkg/booktabs
\usepackage{booktabs}%OK

% Better support for ragged left and right. Provides the commands \RaggedRight and \RaggedLeft.
% Standard LaTeX commands are \raggedright and \raggedleft
% http://www.ctan.org/pkg/ragged2e
\usepackage{ragged2e}

% Create function plots directly in LaTeX
% CTAN: http://www.ctan.org/pkg/pgfplots
\usepackage{pgfplots}
\pgfplotsset{compat=1.11}

%Create appendix
\usepackage[toc, page, titletoc]{appendix}

\makeatletter
\renewcommand{\@chap@pppage}{%
  \clear@ppage
  \thispagestyle{plain}%
  \if@twocolumn\onecolumn\@tempswatrue\else\@tempswafalse\fi
  \null\vfil
  \markboth{}{}%
  {\centering
   \interlinepenalty \@M
   \fontsize{40}{48}\selectfont 
   \sffamily
   \textbf{
   \textcolor{myColorMainA} {\appendixpagename}} \par}%
  \if@dotoc@pp
    \addappheadtotoc
  \fi
  \vfil\newpage
  \if@twoside
    \if@openright
      \null
      \thispagestyle{empty}%
      \newpage
    \fi
  \fi
  \if@tempswa
    \twocolumn
  \fi
}
\makeatother

% Create table of contents for each chapter
% CTAN: http://www.ctan.org/pkg/minitoc
\usepackage{minitoc}
\setcounter{minitocdepth}{2}

% Create nice bibliography
% CTAN: http://www.ctan.org/pkg/biblatex
\usepackage[backend=biber, bibstyle=authoryear, citestyle=authoryear-comp, sortcites, doi=true, dashed=false, indexing, backref=true, sorting=nyt, maxcitenames=1, maxbibnames=100, uniquelist=false, firstinits=true]{biblatex}%refsection=chapter, url=true, 
\setlength\bibitemsep{\baselineskip}
\DeclareNameFormat{coloured-family-given/given-family}{%
  \ifnumequal{\value{listcount}}{1}
    {\textcolor{myColorMainA}{%
     \ifgiveninits
       {\usebibmacro{name:family-given}
         {\namepartfamily}
         {\namepartgiveni}
         {\namepartprefix}
         {\namepartsuffix}}
       {\usebibmacro{name:family-given}
         {\namepartfamily}
         {\namepartgiven}
         {\namepartprefix}
         {\namepartsuffix}}%
     \ifboolexpe{%
       test {\ifdefvoid\namepartgiven}
       and
       test {\ifdefvoid\namepartprefix}}
       {}
       {\usebibmacro{name:revsdelim}}}}
    {\ifgiveninits
       {\usebibmacro{name:given-family}
         {\namepartfamily}
         {\namepartgiveni}
         {\namepartprefix}
         {\namepartsuffix}}
       {\usebibmacro{name:given-family}
         {\namepartfamily}
         {\namepartgiven}
         {\namepartprefix}
         {\namepartsuffix}}}%
  \usebibmacro{name:andothers}}
\DeclareNameAlias{sortname}{coloured-family-given/given-family}
\AtEveryCite{\color{myColorMainA}}
\usepackage{guit}
\urlstyle{sf}

%Create list of abbreviations and glossary
\usepackage[acronym, toc]{glossaries-extra}
\setabbreviationstyle[acronym]{long-short}
%\newglossarystyle{mystyle}{%
   % \setglossarystyle{index}% base this style on the list style
 	%\setabbreviationstyle[acronym]{long-short}
    %
    %
    %
    %\renewcommand*{\glossentry}[2]{%
   % \item\color{myColorMainA}\parbox[t]{3cm}{\bfseries{[\glsentryitem{##1} %
    %\glstarget{##1}{\glossentryname{##1}}]}} \hspace{\stretch{1}} \color{black}\parbox[t]{6.8cm}{\glossentrydesc{##1}\glspostdescription} \hspace{\stretch{1}} \parbox[t]{4.5cm}{see p.\space ##2} \\}%
%}


%Create analytical index
\usepackage{makeidx}
\DeclareIndexFieldFormat{indextitle}{}{}{}

%for schemes and pictures
\usepackage{tikz}
\usetikzlibrary{calc}

%Auxiliary packages for styling (from Nils)
\usepackage[protrusion=true,expansion=true]{microtype}
\usepackage[color=yellow,linecolor=white,textsize=footnotesize]{todonotes} % 
\usepackage{nicefrac}
\usepackage[binary-units=true]{siunitx}
\sisetup{per-mode = symbol-or-fraction}
\sisetup{detect-family = true}

%for nice tables
\usepackage{array}
\newcolumntype{P}[1]{>{\centering\arraybackslash}p{#1}}
\newcolumntype{M}[1]{>{\centering\arraybackslash}m{#1}}

\usepackage{longtable}
\usepackage{rotating}
\usepackage{tabularx}
\usepackage{makecell}

\newcolumntype{s}{>{\hsize=.5\hsize}X}

\newcommand{\heading}[1]{\centering{\textbf{#1}}}

%%%Text specific features, like \textdegree{}
\usepackage{textcomp} 