% Chapter without numbering but with appearance in the Table of Contents
% \addchap is a command from KOMA-Script
%\addchap{Abstract}
\chapter*{Abstract}


Ion beam therapy is a promising technique in cancer treatment because of the ion defined range and favorable dose delivery features. Strict and precise treatment planning and monitoring are now key points for the method developments and full exploitation. In particular, with the aim of optimizing the ion treatment effectiveness, the ion range monitoring is mandatory: different solutions have been explored, but an online treatment check is still a challenge. 
The ion beam treatment monitoring is mainly performed by means of secondary charged or neutral particles. In this context, the detection of the prompt-gammas (PG) emitted during treatments has proven its potential in the ion range control in real time. Since the first evidence of the existing correlation between the emitted gamma profile fall-off and the Bragg peak position, several groups are involved in research activities in order to develop and optimize instruments and methods with the aim of improving this monitoring technique.  Among the others, collimated and Compton cameras are being studied and optimized for this application. The same detectors can also be employed in nuclear medicine for the detection of the radioactive elements decay products.      



A collaboration of 5 institutions in France is involved in the parallel development of two composite detectors for ion beam monitoring and nuclear medicine application, and this thesis is carried out within this collaboration with the detectors clinical trial as final aim.



The project started a few years ago and is now at the final stage. The two cameras have been designed according to simulation studies, and the different components are now under tests.
The collimated camera is composed by a multi-slit tungsten mechanical collimator, set in front of an absorber composed of 96 BGO blocks, for a total size of 380x380x30 mm$^{3}$; each block presents a streaked surface with a 8x8 pixel matrix and the signal is read-out by 4 photomultipliers. A $\sim$3 ns time resolution can be achieved for the prompt gamma detection. The same absorber is part of the Compton camera, in addition to a scatterer section composed by 7 Double-Sided Silicon Detectors 96x96x2 mm$^{3}$ each.
With the collimated camera, the parallel emitted photons are selected by the collimator and a mono-dimensional emission profile can be reconstructed. The Compton camera has a more efficient detection technique, being absent a mechanical collimation system, and could potentially lead to 3D information thanks to the reconstruction of the Compton cone. 
In both cases an additional detector component is needed to temporally tag the incoming beam ions and help rejecting the relevant background (mostly due to neutrons) which strongly affects the prompt gamma yield. A scintillating fiber tagging hodoscope is then under development: it is composed by 128x2 perpendicular scintillating fibers, read-out from both sides by 8 64-channel silicon photomultipliers by Hamamatsu. 
The thesis work consists in the critical evaluation, characterization and tuning of the different components, together with the associated electronics, and of the complete detectors on beam. In parallel, simulation studies can improve the detection technique and optimize the detector structure, as well as pave the way for further applications.    