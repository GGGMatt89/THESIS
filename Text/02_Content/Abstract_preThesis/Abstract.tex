% Chapter without numbering but with appearance in the Table of Contents
% \addchap is a command from KOMA-Script
%\addchap{Abstract}
\chapter*{Abstract}

\glsresetall


The application of nuclear and particle physics techniques in the field of medical diagnosis and pathology treatment is nowadays well-established in the clinical routine. In particular, several medical imaging techniques are based on the exploitation of elementary particles (\glsdesc{pet} (\gls{pet}), \glsdesc{spect} (\gls{spect}), \glsdesc{ct} (\gls{ct}) scans,  etc.), as well as treatment methods, mainly concerning cancer, which causes about 9 millions deaths per year all over the world.
     
In this context, ion beam therapy is a promising technique in cancer treatment because of the ion defined range and favorable dose delivery features with respect to standard photon radiotherapy. Strict and precise treatment planning and monitoring are now key points for the method developments and full exploitation. In particular, with the aim of optimizing the ion treatment effectiveness, the ion range monitoring is mandatory: different solutions have been explored, but an online treatment check is still a challenge. 
The ion beam treatment monitoring is mainly performed by means of secondary charged or neutral particles. In this context, the detection of the prompt-gammas (PG) emitted during treatments has proven its potential in the ion range control in real time. Since the first evidence of the existing correlation between the emitted gamma profile fall-off and the Bragg peak position, several groups are involved in research activities in order to develop and optimize instruments and methods with the aim of improving this monitoring technique.  Among the others, collimated and Compton cameras are being studied and optimized for this application. The same detectors can also be employed in nuclear medicine for the detection of the radioactive elements decay products.      

A collaboration of 4 laboratories in France, called \textit{\glsdesc{clarys}} (\gls{clarys}), is involved in the parallel development of two composite detectors for ion beam monitoring and nuclear medicine applications, and this thesis is carried out within this collaboration with the detectors clinical trial as final aim.

The development project started a few years ago and is now at the final stage. The two cameras have been designed according to simulation studies, and the different components are now under tests.
The collimated camera is composed of a multi-slit tungsten mechanical collimator, set in front of an absorber composed of 30 \glsdesc{bgo} (\gls{bgo}) blocks, for a total size of 210$\times$175$\times$30~mm$^{3}$; each block presents a streaked structure with a 8$\times$8 pseudo-pixel matrix and the signal is read-out by 4 photomultipliers. A $\sim$3~ns time resolution can be achieved on average for the prompt-gamma detection. The same absorber is part of the Compton camera, in addition to a scatterer section composed of 7 \glsdesc{dssd}s (\glspl{dssd}) 96$\times$96$\times$2~mm$^{3}$ each.
With the collimated camera, the parallel emitted photons are selected by the collimator and a mono-dimensional emission profile can be reconstructed. The Compton camera has a more efficient detection technique, thanks to the absence of a mechanical collimation system, and could potentially lead to 3D information via the reconstruction of the Compton cones. These features make it suitable for the application in nuclear medicine, in particular as an alternative to the present \gls{spect} collimated cameras, allowing for accurate and efficient image reconstructions with the usage of high energy gamma source, which should reduce image blurring effects due to attenuation in the patient and the total released dose with respect to the present clinical routine.
 
Concerning the monitoring of ion beam therapy treatments, an additional detector component is needed to temporally and spatially tag the incoming beam ions and help rejecting the relevant background (mostly due to neutrons) which strongly affects the prompt-gamma yield. A scintillating fiber tagging hodoscope, which can be coupled to both collimated and Compton camera, is under development: it is composed of 2 perpendicular planes of 128 scintillating fibers, read-out from both sides by 8 64-channel photomultipliers by Hamamatsu.
 
The thesis work consists in the critical evaluation, characterization and tuning of the different components, together with the associated electronics, and of the complete detectors on beam. In parallel, simulation studies can improve the detection technique and optimize the detector structure, as well as pave the way for further applications.

After a general introduction devoted to expose the thesis context in chapter~\ref{chap::1}, an overview of the instrumental and technical state of the art of the gamma cameras is given in chapter~\ref{chap::2}. Chapter~\ref{chap::3} focuses on the two cameras developed by the \gls{clarys}; the camera components are described in details, and all the characterization measurements performed during the three years of my PhD thesis are explained. Chapter~\ref{chap::4} and~\ref{chap::5} presents the simulation studies I performed with the aim of investigating the potential of the developed detectors for the application on ion beam therapy monitoring and nuclear medicine, respectively. The entire chapter~\ref{chap::6} is dedicated to the description of the tests performed on proton beams for the detector characterization measurements. The final chapter~\ref{chap::7} is used to summarize and discuss all the results obtained in this thesis work; furthermore, the perspectives of the project are fixed on a time-line for the next future, and new research directions emerging from the obtained results are proposed.        