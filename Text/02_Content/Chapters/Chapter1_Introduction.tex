\chapter{Context}\label{chap::1}

\vfill

\minitoc

\newpage

\glsresetall

The application of physics concepts and techniques to the field of health-care is nowadays well-established in the clinical routine. Even if physical techniques have been used in medicine from the earliest time\parencite{Duck2014}, the discipline today known as \enquote{medical physics} emerged and grown in the past century thanks to the increasing knowledge and use of ionizing radiations both for diagnosis and disease treatment. In the late \nth{19} century, the x-ray discovery by R\"{o}ntgen, the radioactivity discovery by Henri Becquerel and the radium and radioactive isotopes studies by Pierre and Marie Curie paved the way to the whole medical physics practice of the next century, where x-ray imaging and radiotherapy were soon established. Following the first successes, more attention were given to radio-protection and dosimetry studies, and the investigations focused on new ways to use radioactive tracers for imaging purpose; this finally led to the birth of nuclear medicine, with the clinical use of the radioisotope \gls{ind131} in 1939~\parencite{Kereiakes1987}. Nuclear medicine rapidly gained importance in the diagnosis clinical panorama, also thanks to the introduction of new detectors, such as the Anger camera introduced in the '60s, and new imaging techniques, such as the detection of photons from positron annihilation (\gls{pet}). Moreover, several alternative methods were proposed or implemented for diagnosis (\gls{mri}, ultrasounds, etc.) and more refined treatment techniques were introduced with the development of particle accelerators after World War 2~\parencite{Keevil2012}.  

At present days, the medical physics progress strongly relies on technological development and computer science, which already revolutionized several fields of science. In particular, cancer research is now a key area for technical and technological development, concerning both diagnosis and treatment\parencite{Webb2009}. The last two decades saw important improvements in radiotherapy techniques and machines, with more precise dose planning and delivery, which strongly reduced the impact of radiations on healthy tissues. This was possible also thanks to refined imaging technologies, allowing for an accurate tumor volume definition, as well as for image-guided treatments. In this scenario, ion beam therapy, already proposed in the middle of the past century, is rapidly spreading thanks to novel beam delivery technologies, able to relatively reduce the treatment costs and allowing for a commercial diffusion of this treatment method (which is still very limited with respect to standard radiotherapy). Notwithstanding the remarkable steps forward disclosed in the last years, there is still wide room for improvements in this fields, which mainly requires strong imaging basis in order to fully profit of the treatment technique potential.

The work presented in this document have been carried out in this context and deals with the development of gamma imaging detectors to be applied in the field of quality assurance for ion beam treatment. Furthermore, the same detectors have been applied to the nuclear medicine field in simulation, with the aim of assessing its possible clinical implementation for a future development of the nuclear medicine clinical routine. 

In the following sections, a general overview of the two main domains concerned in this thesis work is given, with particular focus on the present status of the involved instrumentation.       
  

\section{Ion beam therapy}

Radiation treatment is an essential component of the tumor therapy, being the second most applied and successful kind of therapy after surgery~\parencite{Schardt2010}. The most of the patients treated with radio-therapy techniques receives standard photon treatments, with x-rays coming from linear electron accelerators, while a small percentage undergoes specialized gamma treatments like brachytherapy or gamma knife irradiations. About 1~\% of the radiotherpay patients are irradiated with charged particle beams, with the so-called ion beam therapy~\parencite{Durante2016}.
Ion beam therapy, or \enquote{hadrontherapy}, is a cancer radiation treatment method based on light ion beams instead of photons. It was first proposed by Wilson in 1946~\parencite{Wilson1946}; the author highlighted the physical principles and the possible benefits driven by the implementation of such a kind of radiation in clinical treatments, with particular focus on the simple case of protons and some considerations about heavier ions, like alpha particles. At that time, the accelerator technologies were still under development and the treatment required beam energy were about to be reached; the first patients have been treated almost ten years later in the USA by Lawrence and Larson~\parencite{Tobias1955, Tobias1958} with protons and, later on, with He ions~\parencite{Halperin2006}. Nowadays clinical adapted machines are well-established on the market, making ion beam therapy emerging as a wide-spread technique in the every-day cancer treatment routine all over the world. An intense research effort has been dedicated to this field in the last decades; in addition to considerable improvements achieved in the accelerator technologies, new and refined imaging techniques allowed for important enhancement in the treatment planning, also supported by the continuous development of computer science and the growth of calculation power. In parallel, the biological implications of ion irradiation has been deeply investigated~\parencite{Tobias1982, Brahme2004, Friedrich2012, }. Moreover, since more and more patient are treated every year with this technique, more clinical data are at present available for further study and the connection between physicists and physicians is strongly progressing both in the research field and in treatment practice. Several advancements are expected in the next years, following the extensive research work carried out by several groups in the world. In the following paragraph, the basic physical principles and features of this treatment method are explained, and advantages and drawbacks with respect to standard radiotherapy techniques are analyzed. After that, the need of ion range verification is discussed and detailed in order to reach the main topic for this thesis: the prompt-gamma detection.   

\subsection{Physics}
High-energy beams of charged nuclear particles show a characteristic depth-dose distribution in matter which makes them suitable for the application in cancer treatment as a valid alternative to standard photon therapy (x-ray or megavolt beams), bringing several advantages to the patient side. The peculiar energy deposition profile (\enquote{Bragg curve}) is named for Sir William Henri Bragg, who investigated the slowing-down process of $\mathrm{\alpha}$ particles in air~\parencite{Bragg1904, Bragg1905}. Low-energy (x-rays) and high-energy photons interacting with the patient body lose their energy according to an exponential attenuation law, with the deposited dose decreasing at increasing depth. Even if high-energy photons can spare the entrance surface thanks to a dose peak shift of a few centimeters (mainly due to forward scattered Compton electrons), a high relative dose is delivered to the tissues along the whole beam path. In order to maximize the tumor volume-to-healthy tissue dose ratio, a standard photon treatment always foresees several irradiation fields from different entrance points and angles. In contrast to photons, the energy deposited per unit track increases for increasing depth for protons and heavier ions; the depth-dose profile is characterized by an entrance low relative dose \textit{plateau} and by a narrow high deposited dose peak at the end of the ion range (in the last few millimeters), called \enquote{Bragg peak}. A visual comparison of low- and high-energy photon, proton and carbon ion relative dose profile in water is given in figure~\ref{chap1::fig::Depth-doseProf}. 

\begin{figure}[!htbp]
\centering
\includegraphics[width=0.6\textwidth]{03_GraphicFiles/chapter1_Introduction/depthDoseProf.pdf}
\caption{Relative dose as a function of the particle depth in water for photons at different energies, protons, and carbon ions.}
\label{chap1::fig::Depth-doseProf}
\end{figure} 

The peculiar configuration of the charge particle energy deposition in matter is governed by the energy loss rate formula attributed to Bethe~\parencite{Bethe1930} and Bloch~\parencite{Bloch1933}, often referred as Bethe-Bloch formula, reported in equation~\ref{chap1::eq::bethe-bloch} in its form independent of the mass density. This expression is also known as mass stopping power.\\
  
\begin{equation}
\frac{S}{\rho} = -\frac{dE}{\rho dx} = 4\piN_{A}r^{2}_{e}m_{e}c^{2}\frac{z^{2}}{\beta^{2}}\frac{Z}{A}\bigg[\ln{\frac{2m_{e}c^{2}\beta^{2}\gamma^{2}}{I}}-\beta^{2}-\frac{\delta}{2}-\frac{C}{Z}\bigg]
\label{chap1::eq::bethe-bloch}
\end{equation}

where $N_{A}$ is the Avogadro's number (6.022 $\times$10$^{23}$mol$^{-1}$), $r_{e}$ is the classical electron radius, $m_{e}$ is the electron mass, $z$ is the charge of the projectile, $Z$ and $A$ are the atomic number and weight of the target material, respectively, $c$ is the speed of light, $\beta = v/c$ is the projectile velocity, $\gamma = (1-\beta^{2})^{-1/2}$, $I$ is the mean excitation potential of the target material. The last two terms represent corrections for high energies ($\delta$ term) and low energies ($C$ term) incident ions. As highlighted in~\cite{Newhauser2015}, the two main parameters governing the projectile energy loss rate in the human body are its velocity (energy) and the the material composition; the density in a patient can vary by almost three orders of magnitude, ranging from the air cavity in the lungs to the most dense bones. 

The energy loss equation directly leads to the definition of the ion beam range in matter, which is the integral over the incident energy of the energy loss per track unit (assuming a mono-dimensional path and a continuous energy deposition). To be noticed that the range is not a deterministic value, but it is intended as an average value and defined for the whole beam, not for single incident particles, which are affected by statistical fluctuations in the energy loss, leading to the so-called range straggling (described by different theoretical models, such as the ones in~\cite{Bohr1915, Landau1944, Vavilov1957}). As realized by Bragg and Kleeman~\parencite{Bragg1905}, the range dependence on the incident particle energy can be simply expressed with the power law in equation~\ref{chap1::eq::rangePowerLaw}

\begin{equation}
R(E) = \alpha E^{p}
\label{chap1::eq::rangePowerLaw}
\end{equation}

where the constant $\alpha$ depends on the target material and the constant $p$ is related to the projectile energy (or velocity). It is clear from this formula that the ion range prediction is affected by uncertainties due to both the target and the projectile feature knowledge: in particular, target composition, mass density and linear stopping power, as well as the beam energy distribution, results in a considerable spread in the beam effective range with respect to the predicted one. This topic is more detailed in the following sections.


\subsection{Advantages and drawbacks}

\subsection{Range verification}

\subsection{Secondary radiation techniques}

\subsubsection{Positron Emission Tomography}
\gls{pet} is at present the only method clinically implemented for ion range verification~\parencite{Hishikawa2002, Enghardt2004, Parodi2007, Bauer2013}. \gls{pet} techniques are based on the detection of the two back-to-back 511~keV photons produced by the annihilation of positrons (created by the emitter fragments of nuclear reactions) with patient electrons, resulting in a delayed radiation which should be detected with time coincidences, allowing for an intrinsic background reduction. Nevertheless, the monitoring with positron emitters secondary signal must deal with a limited count rate compared to medical imaging PET, with the lifetime of emitters providing a delayed information that implies the signal integration over a whole treatment fraction (not a single spot or group of spots), with physiological washout effects depending on to the emitters lifetime.

Even if the only available and functional range monitoring system in a clinical center is based on this technique~\parencite{Enghardt2004}, several clinical experience with commercial or adapted PET system already shown intrinsic limitations mainly connected to the ring geometry (not directly applicable to the treatment monitoring due to the presence of the beam) or in general to geometrical constraints limiting the field of view and the resulting system global efficiency and spatial accuracy (the limited detection angle generates artifacts in the final image)~\parencite{Parodi2016}. The research is ongoing and new results are expected for the next years thanks to the introductions of new systems with adapted geometries, to the improvements in acquisition and reconstruction techniques and to the clinical introduction of time-of-flight systems, intrinsically able to improve the detector spatial resolution via interaction time information, and depth-of-interaction reconstruction, which will allow for a more precise spatial reconstruction for reduced angular artifacts effects.

\subsection{Prompt-gammas: physics and features}



\subsection{State of the art of range verification and prompt-gamma techniques}


\section{Nuclear medicine}

\subsection{PET and SPECT}

\subsection{Comparison, advantages and drawbacks}

\subsection{State of the art of SPECT}


\clearpage
%\printbibliography[heading=subbibintoc]
