\chapter{Context}\label{chap::1}

\vfill

\minitoc

\newpage


\glsresetall

\section{Ion beam therapy}

\subsection{Physics}

\subsection{Advantages and drawbacks}

\subsection{Range verification}

\subsection{Secondary radiation techniques}

\subsubsection{Positron Emission Tomography}
PET techniques are based on the detection of the two back-to-back 511~keV photons produced by the annihilation of positrons (created by the emitter fragments of nuclear reactions) with patient electrons, resulting in a delayed radiation which should be detected with time coincidences, allowing for an intrinsic background reduction. Nevertheless, the monitoring with positron emitters secondary signal must deal with a limited count rate compared to medical imaging PET, with the lifetime of emitters providing a delayed information that implies the signal integration over a whole treatment fraction (not a single spot or group of spots), with physiological washout effects depending on to the emitters lifetime.

Even if the only available and functional range monitoring system in a clinical center is based on this technique~\parencite{Enghardt2004}, several clinical experience with commercial or adapted PET system already shown intrinsic limitations mainly connected to the ring geometry (not directly applicable to the treatment monitoring due to the presence of the beam) or in general to geometrical constraints limiting the field of view and the resulting system global efficiency and spatial accuracy (the limited detection angle generates artifacts in the final image)~\parencite{Parodi2016}. The research is ongoing and new results are expected for the next years thanks to the introductions of new systems with adapted geometries, to the improvements in acquisition and reconstruction techniques and to the clinical introduction of time-of-flight systems, intrinsically able to improve the detector spatial resolution via interaction time information, and depth-of-interaction reconstruction, which will allow for a more precise spatial reconstruction for reduced angular artifacts effects.

\subsection{Prompt-gammas: physics and features}



\subsection{State of the art of range verification and prompt-gamma techniques}


\section{Nuclear medicine}

\subsection{PET and SPECT}

\subsection{Comparison, advantages and drawbacks}

\subsection{State of the art of SPECT}


\clearpage
%\printbibliography[heading=subbibintoc]
